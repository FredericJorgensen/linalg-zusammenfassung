%%% SYMBOLS USED BY NUMBERINGS, ENVIRONMENTS, ... %%%%%%%%%%%%%%%%%%%%%%%%%%%%%%

% \renewcommand*\qedsymbol{$\blacksquare$} % alternative QED symbol
\renewcommand{\thefootnote}{\arabic{footnote}} % normal footnotes on page
\renewcommand{\thempfootnote}{\fnsymbol{mpfootnote}} % footnotes on minipages, e.g. in mdframed environments

%%% MDFRAMED STYLES %%%%%%%%%%%%%%%%%%%%%%%%%%%%%%%%%%%%%%%%%%%%%%%%%%%%%%%%%%%%

% thick frame and bar for title

\mdfdefinestyle{style-box}{
  linewidth=1pt,
  linecolor=Gray!20,
%   roundcorner=3pt,
  innerleftmargin=0.4\baselineskip,
  innerrightmargin=0.2\baselineskip,
  innertopmargin=0.2\baselineskip,
  innerbottommargin=0.2\baselineskip,
  frametitlebackgroundcolor=Gray!20,
  frametitleaboveskip=0.2pt,
  frametitlebelowskip=0.2pt,
  theoremseparator=,
  theoremspace=\hfill,
  theoremtitlefont=\mdseries\scshape,
  nobreak=true
}

% highlighted background

\mdfdefinestyle{style-background}{
  hidealllines=true,
  backgroundcolor=Gray!5,
  innerleftmargin=0.4\baselineskip,
  innerrightmargin=0.2\baselineskip,
  innertopmargin=0.2\baselineskip,
  innerbottommargin=0.2\baselineskip,
}

% thin frame

\mdfdefinestyle{style-thinframe}{
  linewidth=0.4pt,
  linecolor=Gray!30,
  innerleftmargin=0.5\baselineskip,
  innerrightmargin=0.5\baselineskip,
  innertopmargin=0.4\baselineskip,
  innerbottommargin=0.4\baselineskip,
}

%%% ENVIRONMENTS %%%%%%%%%%%%%%%%%%%%%%%%%%%%%%%%%%%%%%%%%%%%%%%%%%%%%%%%%%%%%%%

% Definition

\mdtheorem[
  style=style-box,
  linecolor=color-definition,
  frametitlebackgroundcolor=color-definition
]{definition}{\definitionname}[section]

% Theorem

\mdtheorem[
  style=style-box,
  linecolor=color-theorem,
  frametitlebackgroundcolor=color-theorem,
  font=\itshape
]{theorem}{\theoremname}[section]

% Proposition

\mdtheorem[
  style=style-box,
  linecolor=color-proposition,
  frametitlebackgroundcolor=color-proposition,
  font=\itshape
]{proposition}[theorem]{\propositionname}

% Corollary

\mdtheorem[
  style=style-box,
  linecolor=color-corollary,
  frametitlebackgroundcolor=color-corollary,
  font=\itshape
]{corollary}[theorem]{\corollaryname}

% Lemma

\mdtheorem[
  style=style-box,
  linecolor=color-lemma,
  frametitlebackgroundcolor=color-lemma,
  font=\itshape
]{lemma}[theorem]{\lemmaname}

\theoremstyle{remark}

% Remark

\newtheorem*{remark}{\remarkname}
\surroundwithmdframed[
  style=style-background,
  backgroundcolor=color-remark
]{remark}

% Enumeration Remark

\newenvironment{enumremark}{
  \begin{remark}
    \begin{enumerate}[(a)]
      \item[]
}{
    \end{enumerate}
  \end{remark}
}

% Example

\newtheorem*{example}{\examplename}
\surroundwithmdframed[
  style=style-background,
  backgroundcolor=color-example
]{example}

% Proof

\surroundwithmdframed[
  style=style-thinframe
]{proof}

%%% TEXT FORMATTING %%%%%%%%%%%%%%%%%%%%%%%%%%%%%%%%%%%%%%%%%%%%%%%%%%%%%%%%%%%%

% definitions

\newcommand*{\df}[1]{\colorbox{color-definition}{\emph{#1}}}
