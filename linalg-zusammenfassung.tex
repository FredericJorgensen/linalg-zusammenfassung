%%% DOCUMENT TYPE %%%%%%%%%%%%%%%%%%%%%%%%%%%%%%%%%%%%%%%%%%%%%%%%%%%%%%%%%%%%%%

\documentclass[11pt, a4paper]{article}

%%% SETUP %%%%%%%%%%%%%%%%%%%%%%%%%%%%%%%%%%%%%%%%%%%%%%%%%%%%%%%%%%%%%%%%%%%%%%

% Analysis


\usepackage{mathtools}
\usepackage{amsfonts} 
\usepackage{fancybox}
\usepackage{textcomp }
\usepackage{bbold}
\usepackage{amssymb}
\usepackage{titlesec}
\usepackage[english,ngerman]{babel}
\usepackage{mathabx} 

\title{Lineare Algebra II}
\author{Frederic Jørgensen}
\date{Erstellt: 01. Februar 2020, Zuletzt ge\aee ndert: \today \\                           % deutsche Ausgabe
	 \selectlanguage{ngerman}}
\pagestyle{headings}
\setlength{\parindent}{0pt}


\newcommand{\sectionbreak}{\clearpage} %Seite bei jedem Abschnitt wechs^{•}eln


%%% PACKAGES %%%%%%%%%%%%%%%%%%%%%%%%%%%%%%%%%%%%%%%%%%%%%%%%%%%%%%%%%%%%%%%%%%%

% Encoding

\usepackage[utf8]{inputenc}
\usepackage[T1]{fontenc}

% Geometry

\usepackage{geometry} % edit margins of paper
\usepackage{setspace} % edit line spacing
\usepackage{fancyhdr} % header, footer
\usepackage{titlesec} % edit format of titles

% Visual

\usepackage[dvipsnames]{xcolor} % colors
\usepackage{tikz} % graphics
\usepackage[framemethod=tikz]{mdframed} % frames, better theorems

% Math

\usepackage{amsmath} % math tools
\usepackage{amssymb} % math symbols
\usepackage{amsthm} % thereoms
\usepackage{mathtools} % math tools

% Referencing

\usepackage{nameref}
\usepackage{hyperref}
\usepackage{cleveref}

% Useful

\usepackage[shortlabels]{enumitem} % enumerations

% Other

\usepackage{lastpage} % get number of last page

%%% MARGINS %%%%%%%%%%%%%%%%%%%%%%%%%%%%%%%%%%%%%%%%%%%%%%%%%%%%%%%%%%%%%%%%%%%%

\geometry{a4paper, left=20mm, right=20mm, top=20mm, bottom=20mm, includehead}

%%% COLORS %%%%%%%%%%%%%%%%%%%%%%%%%%%%%%%%%%%%%%%%%%%%%%%%%%%%%%%%%%%%%%%%%%%%%

%%% COLOR DEFINITIONS %%%%%%%%%%%%%%%%%%%%%%%%%%%%%%%%%%%%%%%%%%%%%%%%%%%%%%%%%%

\colorlet{color-definition}              {Blue!20}%{SpringGreen!20}
\colorlet{color-theorem}                {Brown!25}%{Apricot!13}
\colorlet{color-proposition}            {ProcessBlue!13}% {Apricot!13}
\colorlet{color-corollary}              {Salmon!12}%{Apricot!13}
\colorlet{color-lemma}                  {Brown!7}%{Apricot!13}
\colorlet{color-remark}                 {Gray!4}
\colorlet{color-example}                {Lavender!7}
% \colorlet{color-proof}                  {FILL COLOR HERE}


%%% CAPTIONS %%%%%%%%%%%%%%%%%%%%%%%%%%%%%%%%%%%%%%%%%%%%%%%%%%%%%%%%%%%%%%%%%%%

%%% CAPTION DEFINITION %%%%%%%%%%%%%%%%%%%%%%%%%%%%%%%%%%%%%%%%%%%%%%%%%%%%%%%%%

\newcommand*{\definitionname}{Definition}
\newcommand*{\theoremname}{Theorem}
\newcommand*{\propositionname}{Proposition}
\newcommand*{\corollaryname}{Corollary}
\newcommand*{\lemmaname}{Lemma}
\newcommand*{\remarkname}{Remark}
\newcommand*{\examplename}{Example}


%%% LANGUAGE %%%%%%%%%%%%%%%%%%%%%%%%%%%%%%%%%%%%%%%%%%%%%%%%%%%%%%%%%%%%%%%%%%%

% load language setup (may also include loading packages)

%%% SETUP %%%%%%%%%%%%%%%%%%%%%%%%%%%%%%%%%%%%%%%%%%%%%%%%%%%%%%%%%%%%%%%%%%%%%%

\usepackage[english]{babel}

%%% CAPTION REDEFINITION %%%%%%%%%%%%%%%%%%%%%%%%%%%%%%%%%%%%%%%%%%%%%%%%%%%%%%%



%%% HYPHENATION %%%%%%%%%%%%%%%%%%%%%%%%%%%%%%%%%%%%%%%%%%%%%%%%%%%%%%%%%%%%%%%%



%%% TITLES %%%%%%%%%%%%%%%%%%%%%%%%%%%%%%%%%%%%%%%%%%%%%%%%%%%%%%%%%%%%%%%%%%%%%

\setcounter{secnumdepth}{2}

\titleformat{\section}[block]
{\normalfont\Large\bfseries}{\thesection}{1em}{}
\titleformat{\subsection}[block]
{\normalfont\large\bfseries}{\thesubsection}{1em}{}
\titleformat{\subsubsection}[block]
{\normalfont\normalsize\bfseries}{\thesubsubsection}{1em}{}

\titlespacing*{\section}{0pt}{3.25ex plus 1ex minus .2ex}{2.0ex plus .2ex}
\titlespacing*{\subsection}{0pt}{2.75ex plus 1ex minus .2ex}{1.0ex plus .2ex}
\titlespacing*{\subsubsection}{0pt}{2.25ex plus 0.8ex minus 0.1ex}{0.0ex plus .2ex}

%%% SPACING, LENGTHS, INDENTATION %%%%%%%%%%%%%%%%%%%%%%%%%%%%%%%%%%%%%%%%%%%%%%

\setstretch{1.05} % scaling of space between lines
% \setlength{\topsep}{FILL LENGTH HERE}
% \setlength{\itemsep}{FILL LENGTH HERE}
\setlength{\parskip}{4.0pt plus 1.0pt minus 1.0pt} % space between paragraphs
\setlength{\parindent}{0pt} % indentation of paragraphs

%%% HEADER, FOOTER %%%%%%%%%%%%%%%%%%%%%%%%%%%%%%%%%%%%%%%%%%%%%%%%%%%%%%%%%%%%%

\pagestyle{fancy}
\fancyhf{} % clear everything
\lhead{}
\chead{\large \bfseries \LaTeX{} (Math) Template}
\rhead{Page \thepage /\pageref*{LastPage}}
\lfoot{}
\cfoot{}
\rfoot{}

%%% SHORTCUTS %%%%%%%%%%%%%%%%%%%%%%%%%%%%%%%%%%%%%%%%%%%%%%%%%%%%%%%%%%%%%%%%%%

%%% SINGLE SYMBOLS %%%%%%%%%%%%%%%%%%%%%%%%%%%%%%%%%%%%%%%%%%%%%%%%%%%%%%%%%%%%

% Logic

% \forall exists
% \exists exists
% \lnot exists
% \lor exists
% \land exists
\newcommand*{\limp}{\rightarrow}
\newcommand*{\limps}{\; \limp \;} % \limp with some space around
\newcommand*{\leqv}{\leftrightarrow}
\newcommand*{\leqvs}{\; \leqvs \;} % \leqv with some space around

% Meta Logic

% \implies exists
% \iff exists

% Colon Stuff

\newcommand*{\cl}{\colon}
\newcommand*{\cleq}{\coloneqq}
\newcommand*{\eqcl}{\eqqcolon}

% Sets

\newcommand*{\N}{\mathbb{N}} % natural numbers
\newcommand*{\Z}{\mathbb{Z}} % integers
\newcommand*{\Q}{\mathbb{Q}} % rational numbers
\newcommand*{\R}{\mathbb{R}} % real numbers
\newcommand*{\C}{\mathbb{C}} % complex numbers

%%% MATH OPERATORS %%%%%%%%%%%%%%%%%%%%%%%%%%%%%%%%%%%%%%%%%%%%%%%%%%%%%%%%%%%%%

% General

\DeclareMathOperator{\id}{id}
\DeclareMathOperator{\sgn}{sgn}

%%% TEMPLATES %%%%%%%%%%%%%%%%%%%%%%%%%%%%%%%%%%%%%%%%%%%%%%%%%%%%%%%%%%%%%%%%%%

% General

% write a set definition like: { #1 | #2 }
\newcommand*{\setdefinition}[2]{
  \{#1 \mid #2\}
}

% write a nice map definition
\newcommand*{\mapdefinition}[5]{
  \begin{align*}
    #1 \cl #2 &\to     #3 \\
           #4 &\mapsto #5
  \end{align*}
}


%%% FORMATTING %%%%%%%%%%%%%%%%%%%%%%%%%%%%%%%%%%%%%%%%%%%%%%%%%%%%%%%%%%%%%%%%%

%%% SYMBOLS USED BY NUMBERINGS, ENVIRONMENTS, ... %%%%%%%%%%%%%%%%%%%%%%%%%%%%%%

% \renewcommand*\qedsymbol{$\blacksquare$} % alternative QED symbol
\renewcommand{\thefootnote}{\arabic{footnote}} % normal footnotes on page
\renewcommand{\thempfootnote}{\fnsymbol{mpfootnote}} % footnotes on minipages, e.g. in mdframed environments

%%% MDFRAMED STYLES %%%%%%%%%%%%%%%%%%%%%%%%%%%%%%%%%%%%%%%%%%%%%%%%%%%%%%%%%%%%

% thick frame and bar for title

\mdfdefinestyle{style-box}{
  linewidth=1pt,
  linecolor=Gray!20,
%   roundcorner=3pt,
  innerleftmargin=0.4\baselineskip,
  innerrightmargin=0.2\baselineskip,
  innertopmargin=0.2\baselineskip,
  innerbottommargin=0.2\baselineskip,
  frametitlebackgroundcolor=Gray!20,
  frametitleaboveskip=0.2pt,
  frametitlebelowskip=0.2pt,
  theoremseparator=,
  theoremspace=\hfill,
  theoremtitlefont=\mdseries\scshape,
  nobreak=true
}

% highlighted background

\mdfdefinestyle{style-background}{
  hidealllines=true,
  backgroundcolor=Gray!5,
  innerleftmargin=0.4\baselineskip,
  innerrightmargin=0.2\baselineskip,
  innertopmargin=0.2\baselineskip,
  innerbottommargin=0.2\baselineskip,
}

% thin frame

\mdfdefinestyle{style-thinframe}{
  linewidth=0.4pt,
  linecolor=Gray!30,
  innerleftmargin=0.5\baselineskip,
  innerrightmargin=0.5\baselineskip,
  innertopmargin=0.4\baselineskip,
  innerbottommargin=0.4\baselineskip,
}

%%% ENVIRONMENTS %%%%%%%%%%%%%%%%%%%%%%%%%%%%%%%%%%%%%%%%%%%%%%%%%%%%%%%%%%%%%%%

% Definition

\mdtheorem[
  style=style-box,
  linecolor=color-definition,
  frametitlebackgroundcolor=color-definition
]{definition}{\definitionname}[section]

% Theorem

\mdtheorem[
  style=style-box,
  linecolor=color-theorem,
  frametitlebackgroundcolor=color-theorem,
  font=\itshape
]{theorem}{\theoremname}[section]

% Proposition

\mdtheorem[
  style=style-box,
  linecolor=color-proposition,
  frametitlebackgroundcolor=color-proposition,
  font=\itshape
]{proposition}[theorem]{\propositionname}

% Corollary

\mdtheorem[
  style=style-box,
  linecolor=color-corollary,
  frametitlebackgroundcolor=color-corollary,
  font=\itshape
]{corollary}[theorem]{\corollaryname}

% Lemma

\mdtheorem[
  style=style-box,
  linecolor=color-lemma,
  frametitlebackgroundcolor=color-lemma,
  font=\itshape
]{lemma}[theorem]{\lemmaname}

\theoremstyle{remark}

% Remark

\newtheorem*{remark}{\remarkname}
\surroundwithmdframed[
  style=style-background,
  backgroundcolor=color-remark
]{remark}

% Enumeration Remark

\newenvironment{enumremark}{
  \begin{remark}
    \begin{enumerate}[(a)]
      \item[]
}{
    \end{enumerate}
  \end{remark}
}

% Example

\newtheorem*{example}{\examplename}
\surroundwithmdframed[
  style=style-background,
  backgroundcolor=color-example
]{example}

% Proof

\surroundwithmdframed[
  style=style-thinframe
]{proof}

%%% TEXT FORMATTING %%%%%%%%%%%%%%%%%%%%%%%%%%%%%%%%%%%%%%%%%%%%%%%%%%%%%%%%%%%%

% definitions

\newcommand*{\df}[1]{\colorbox{color-definition}{\emph{#1}}}



%%% DOCUMENT %%%%%%%%%%%%%%%%%%%%%%%%%%%%%%%%%%%%%%%%%%%%%%%%%%%%%%%%%%%%%%%%%%%

%%Todo:
% Definition Integritaetsring


\begin{document}
\maketitle
\tableofcontents
\newpage
\begin{abstract}
	%enterlater
	\centerline{\bt{Stand zum ersten Semester}, Template von Johann Birnick}
\end{abstract}
\section{Eigenwerte}
\subsection{Trigonalisierung}
\begin{definition}
Sei $F : V\ \rightarrow V$ ein Endomorphismus und $W \subset V$ ein Untervektorraum. $W$ heisst $F$-invariant, wenn $F(W) \subset W$.
\end{definition}
\begin{remark}
Ist $W \subset V$ ein $F$-invarianter Unterraum, so ist $P_{F|W}$ ein Teiler von $P_F$.
\end{remark}

\begin{definition}
Eine \bt{Fahne} $(V_r)$ in einem $n$-dimensionalem $V$ ist eine Kette 
\\ \centerline{ $\{0\} = V_0 \subset V_1 \subset ... \subset V_n = V$}
von Untervektorr\aee umen mit $dim V_r = r$. Ist $F \in End(V)$, so heisst die Fahne $F$-invariant, wenn 
\\ \centerline{$F(V_r) \subset V_r$ f\uee r alle $r \in \{0, ... n\}$.}
\end{definition}

\begin{remark}
F\uee r $F \in End(V)$ sind folgende Bedingungen \aee quivalent:
\begin{enumerate}
\item[(i)] Es gibt eine $F$-invariante Fahne in $V$.
\item[(ii)] Es gibt eine Basis $\mathcal{B}$ von V, so dass $M_\mathcal{B}  (F)$ eine obere Dreiecksmatrix ist.
\end{enumerate}
\end{remark}


\begin{theorem} \textbf{(Trigonalisierungssatz)} F\uee r einen Endomorphismus $F$ eines $n$-dimensionalen $K$-Vektorraumes sind folgende Bedingungen \aee quivalent:
\begin{enumerate}
\item[(i)] $F$ ist trigonalisierbar.
\item[(ii)] Das charakteristische Polynom $P_F$ zerf\aee llt in Linearfaktoren.
\end{enumerate}
\end{theorem}

\begin{corollary} Jeder Endomorphismus eines endlich-dimensionalen komplexen Vektorraumes ist trigonalisierbar.
\end{corollary}

\begin{remark} \bt{Rechenverfahren zur Trigonalisierung eines Endomorphismus} von $A \in M(n \times n; K)$
\begin{enumerate}
\item Pr\uee fe, dass $P_A$ in Linearfaktoren zerf\aee llt, aber A nicht diagonalisierbar ist. 
\\
Bestimme einen Eigenvektor $v_1 \in Eig(A, \lambda_1 )$ und erg\aee nze diesen zu einer Basis $\mathcal{B} := (v_1, e_i, ..., e_j)$ des $K^n$. 
\\Betrachte $S^{-1}_1 := T^{\mathcal{B}_1}_{\mathcal{K}}$ und berechne $A_2 = S_1 \cdot A \cdot S_1^{-1} = $
$\left(\begin{array}{ccccc}{
\lambda_{1}} & {*} & {\cdots} & {\cdots} & {*} 
\\ {0} & {\lambda_{2}} & {*} & {\cdots} & {*} 
\\ {\vdots} & {0} & {} & {} 
\\ {\vdots} & {\vdots} & {} & {A_{2}^{\prime}} 
\\ {0} & {0} & {} & {} & {}\end{array}\right)$
\item Bestimme einen Eigenvektor $v_2^{\prime} \in Eig(A_{2}^{\prime}, \lambda_2) $ f\uee r einen Eigenwert $\lambda_2$ von $A_{2}^{\prime}$, erg\aee nze $v_1, v_2$ zu einer Basis $\mathcal{B}_2$ des $K^n$, wobei $v_2$ wie  $v_2^{\prime}$ ist, aber zus\aee tzlich eine 0 als ersten Eintrag hat. Betrachte $S^{-1}_2 := T^{\mathcal{B}_2}_{\mathcal{K}}$ und berechne $A_3 = S_2 \cdot A \cdot S_2^{-1} = $
$\left(\begin{array}{ccccc}{
\lambda_{1}} & {*} & {\cdots} & {\cdots} & {*} 
\\ {0} & {\lambda_{2}} & {*} & {\cdots} & {*} 
\\ {\vdots} & {0} & {} & {} 
\\ {\vdots} & {\vdots} & {} & {A_{3}^{\prime}} 
\\ {0} & {0} & {} & {} & {}\end{array}\right)$

\item Das macht so oft, bis $ D := A_n =S_{n-1} \cdot A \cdot S_{n-1}^{-1}$ eine obere Dreiecksmatrix ist.
\end{enumerate}
\end{remark}
\subsection{Potenzen eines Endomorphismus}
\begin{theorem}
\bt{Satz von Caley-Hamilton} Sei $V$ endlichdimensional und $F \in End(V)$. Dann ist
\\ \centerline{$P_F(F) = 0 \in End(V)$.}
\end{theorem}
\begin{definition}
Eine Teilmenge $\mathcal{I} \subset R$ eines kommutativen Ringes $R$ heisst \bt{Ideal}, wenn gilt:
\begin{enumerate}
\item[I 1] $P, Q \in \mathcal{I} \implies P - Q \in \mathcal{I}$ (Untergruppenkriterium)
\item[I 2] $P \in \mathcal{I}, Q \in R \implies Q \cdot P \in \mathcal{I}$
\end{enumerate}
\end{definition}
\begin{example}
Sei F ein Endomorphismus. Das \bt{Ideal von F} ist definiert als
\\ \centerline{$\mathcal{I}_F := \{ P(t) \in K[t] : P(F) = 0  \} \subset K[t]$.}
\end{example}

\begin{theorem}
Zu jedem Ideal $\mathcal{I} \subset K[t]$ mit $\mathcal{I} \neq \{0\}$ gibt es ein eindeutiges \bt{Minimalpolynom} $M$ mit:
\begin{enumerate}
\item $M$ ist normiert
\item F\uee r jedes $P \in \mathcal{I}$ gibt es ein $Q \in K[t]$ mit $P = Q \cdot M$.
\end{enumerate}
Man nennt das Minimalpolynom von $\mathcal{I}_F$ auch das \bt{Minimalpolynom von F}$M_F$.
\end{theorem}

\begin{theorem}
Sei $n = dim (V)$ und $F \in End(V)$. Dann gilt:
\begin{enumerate}
\item $M_F | P_F$
\item $P_F | M_F^n$
\end{enumerate}
\end{theorem}
%Anderes aus Linalg I ergaenzen?
\begin{definition}
$F \in End(V)$ heisst \bt{nilpotent}, wenn $F^k = 0$ f\uee r ein $k \geq 1$.
\end{definition}
\begin{theorem}
Ist $F \in End(V)$ und $n = dim(V)$. Dann sind folgende Aussagen \aee quivalent:
\begin{itemize}
\item[(i)] $F$ ist nilpotent.
\item[(ii)] $F^d = 0$ f\uee r ein d mit $1 \leq d \leq n$.
\item[(iii)] $P_F = \pm t^n$.
\item[(iv)] Es gibt eine Basis $\mathcal{B}$ von $V$, so dass 
\\ \centerline{
$M_{\mathcal{B}}(F)=\left(\begin{array}{ccc}
{0} & {} & {*} \\ {} & {\ddots} & {} 
\\ {0} & {} & {0}\end{array}\right)$.
}
\end{itemize}
\end{theorem}

\subsection{Die Jordansche Normalform}
\begin{definition}
Sei $F \in End(V)$, so dass $P_F$ in Linearfaktoren zerf\aee llt. Wenn $dim(Eig(F; \lambda_i)) < \mu (P_F; \lambda_i) =: r_i$, kann man die Dimension des Eigenraums vergr\oee ssern: 
\\ \centerline{$Eig(F; \lambda_i) \subset Ker(F - \lambda_i id_v)^{r_i} = Hau(F; \lambda_i)$. }
$Hau(F; \lambda_i)$ nennt man den \bt{Hauptraum} von $F$ zum Eigenwert $\lambda_i$.
\end{definition}
\begin{theorem} \bt{(Satz \uee ber die Hauptraumzerlegung)}
Sei $F \in End_K(V)$, so dass $P_F$ in Linearfaktoren zerf\aee llt. Es sei $V_i := Hau(F, \lambda_i)$ f\uee r alle paarweise verschiedenen Eigenwerte $\lambda_1, ... , \lambda_n \in K$ von $F$. Sei $k := Rg(F)$. Dann gilt:
\begin{enumerate}
\item $F(V_i) \subset V_i$ und $dim(V_i) = \mu (P_F, \lambda_i)$ f\uee r $i = 1, ..., k$. 
\item $V =\bigoplus\limits_{i \in I} V_i$ mit $I=1, ..., k$.
\item $F$ hat eine Zerlegung $F = F_D + F_N$ mit
\begin{enumerate}
\item $F_D$ ist diagonalisierbar
\item$ F_N$ ist nilpotent
\item $F_N$ und $F_D$ kommutieren
\item $F_N$ und $F_D$ lassen sich als Polynome von $F$ schreiben und kommutieren insbesondere mit $F$
\item Wenn man (a), (b) und (c) verlangt, ist diese Zerlegung eindeutig
\end{enumerate}
\end{enumerate}
\end{theorem}

\begin{corollary}
Sei $A \in M(n \times n; K)$, so dass $P_A$ in Linearfaktoren zerf\aee llt. Dann $\exists S \in GL(n)$: 
$$
S A S^{-1}=\left(\begin{array}{cccc}{} & {} & {} & {} 
\\ {\lambda_{1} E_{r_{1}}+N_{1}} & {} & {} & {0} 
\\ {} & {} & {\ddots} & {} 
\\ {0} & {} & {} & {\lambda_{1} E_{r_{1}}+N_{1}} 
\end{array}\right)
$$
wobei f\uee r $i = 1, ... k$
\\ \centerline{$
\lambda_{1} E_{r_{1}}+N_{1} = 
\left(\begin{array}{ccc}
{\lambda_{i}} & {} & {*} 
\\ {} & {\ddots} & {} 
\\ {0} & {} & {\lambda_{i}}\end{array}\right) \in M(r_i \times r_i; K)$.}
mit nilpotenten $N_i$. Insbesondere ist $\tilde{A} = D + N$, wobei $D$ eine Diagonalmatrix und $N$ nilpotent ist und $N$ und $D$ kommutieren.
%%%
\end{corollary}

\begin{lemma}
\bt{(Lemma von Fitting)} Zu einem $G \in End_K(V) $ betrachte
\\ \centerline{$d := min \{ l \in \N : Ker (G^l) =Ker(G^{l + 1}) \}$ und $r := \mu (P_G ; 0)$.}
\\wobei $G^0 := id_V$. Dann gilt:
\begin{enumerate}
\item $d =min \{l : Im(G^l) = Im(G^{l + 1}) \}.$
\item $Ker (G^{d + i}) = Ker (G), \ Im (G^{d + i}) = Im (G^d)\ \forall i \in \N$.
\item Die R\aee ume $U := Ker (G^d)$ und $W := Im (G^d)$ sind $G$-invariant.
\item $(G|U)^d = 0$ und $G|W: W \rightarrow W$ ist ein Isomorphismus.
\item F\uee r das Minimalpolynom von $G|U$ gilt $M_{G|U} = t^d$.
\item $V = U \oplus W$, $dim(U) = r \geq d$, $dim(W) = n - r$.
\\ Insbesondere gibt es eine Basis $\mathcal{B}$ von $V$, so dass
\\ \centerline{$M_\mathcal{B}(G) =
\left(\begin{array}{ll}{N} & {0} 
\\ {0} & {C}\end{array}\right) 
$
mit  $N^d = 0$ und $C \in GL (n-r; K)$.}
\end{enumerate}
\end{lemma}
%to be continued
%Satz seite 256 nochmal aufschreiben
\subsection{Die Normalform von nilpotenten Abbildungen}
\begin{definition}
Die \bt{Jordanmatrix} ist definiert als $J_{k}:=\left(\begin{array}{cccc}
{0} & {1} & {} & {0} 
\\ {} & {\ddots} & {\ddots} & {} 
\\ {} & {} & {\ddots} & {1} 
\\ {0} & {} & {} & {0}\end{array}\right) \in \mathrm{M}(k \times k ; K)$ f\uee r die $J^k = 0$ gilt und $J^{k-1} \neq 0$.
\end{definition}
\begin{theorem}
Sei $G$ ein nilpotenter Endomorphismus eines K-Vektorraums $V$ und $d := min \{l | G^l = 0 \}$. Dann gibt es eindeutig bestimmte Zahlen $s_1, ..., s_d \in \mathbb{N}$ mit 
\\ \centerline{$d \cdot s_d +(d - 1)s_{d-1} + ... + s_l = r = dim V$}
\\und eine Basis $\mathcal{B}$ von $V,$, so dass 
$M_\mathcal{B}(G) =
\left( \begin{array}{cccccccccc}
{J_{d}} & {} & {} & {} & {} & {} & {} & {} & {} & {}
\\ {} & {\ddots} & {} & {} & {} & {} & {} & {} & {0} & {}
\\ {} & {} & {J_{d}} & {} & {} & {} & {} & {} & {} & {}
\\ {} & {} & {} & {J_{d-1}} & {} & {} & {} & {} & {} & {}
\\ {} & {} & {} & {} & {}\ddots & {} & {} & {} & {} & {}
\\ {} & {} & {} & {} & {} & {J_{d-1}} & {} & {} & {} & {}
\\ {} & {} & {} & {} & {} & {} & {\ddots} & {} & {} & {}
\\ {} & {} & {} & {} & {} & {} & {} & {J_{l}} & {} & {}
\\ {} & {0} & {} & {} & {} & {} & {} & {} & {\ddots} & {}
\\ {} & {} & {} & {} & {} & {} & {} & {} & {} & {J_{l}}
\end{array} 
\right)$,
\\wobei $J_d$ $s_d$-mal, $J_{d-1}$ $s_{d-1}$-mal, ... und $J_l$ $s_1$-mal vorkommt.
\end{theorem}

\begin{theorem}
\bt{(Michael-Jordan Normalform)}
Sei $F \in End(V$, wobei das charakteristische Polynom von $F$ in Linearfaktoren zerf\aee llt mit
\\ \centerline{$P_F = \pm (t - \lambda_1)^{r_1} \cdot ... \cdot (t - \lambda_k)^{r_k}$.}
Dann gibt es eine Basis $\mathcal{B}$ von $V$ mit 
$M_\mathcal{B}(F) = 
\left(\begin{array}{ccc}
\lambda_{1} E_{r_{1}}+N_{1} & & 0 
\\ & \ddots &  
\\ 0 & & \lambda_{k} E_{r_{k}}+N_{k}
\end{array}\right).$
\\ $\lambda_i E_{r_i} + N_i = \left(\begin{array}{ccc}
B_{1, d_1} & & 0 
\\ & \ddots &  
\\ 0 & & B_{1, d_2}
\end{array}\right)$
 besteht dabei aus Jordanbl\oee cken, welche diese Form haben:
\\ \centerline{$B_{i, d} = \left(\begin{array}{cccc}\lambda_{i} & 1 & & 
\\ & \ddots & \ddots & 
\\ & & \ddots & 1 
\\ & & & \lambda_{i}\end{array}\right).$}

Ein Jordanblock hat die Gr\oee sse $d$, wenn er d Eintr\aee ge mit $\lambda_i$ und $d - 1$ Einsen oberhalb der Diagonale hat.
Es gilt $1 \leq d_i \leq \mu (M_F, \lambda_i)$ f\uee r $\forall i$.  Der gr\oee sste Jordanblock hat dabei genau die Gr\oee sse $\mu (M_F, \lambda_i)$.

Die Summe der Anzahl der Jordanbl\oee cke multipliziert mit ihrer Gr\oee sse ergibt die algebraische Vielfachheit des Eigenwerts $\lambda_i$.

Man nennt die Elemente $\lambda_i, r_i, d_j$ sowie die Anzahl der Jordanbl\oee cke $s_j^{(i)}$ der Gr\oee sse $d_j$ zum Eigenwert $\lambda_i$ \bt{Invarianten} des Endomorphismus $F$.

Insbesondere ist also die jordansche Normalform ist bis auf die Reihenfolge der Jordanbl\oee cke eindeutig bestimmt.

\textit{Konvention:} Die Jordanbl\oee cke zu einem Eigenwert $\lambda_i$ werden der Gr\oee sse nach absteigend geordnet.
\end{theorem}

\begin{corollary}
$F \in End(V)$ ist genau dann diagonalisierbar, wenn $M_F$ folgende Form hat:
\\ \centerline{$M_F = (t - \lambda_1) \cdot ... \cdot (t - \lambda_k)$,}
wobei $\lambda_1, ... , \lambda_k$ die Eigenwerte von $F$ sind.
\end{corollary}

\begin{remark}
\bt{Rechenverfahren zur Bestimmung einer Jordan-Normalform} von $A \in M(n \times n; K)$
\begin{enumerate}
\item Bestimme die Eigenwerte $\lambda_1, ... \lambda_k$ der Matrix $A$ .
\item Bestimme die Anzahl und Gr\oee sse der Jordanbl\oee cke:

Sei $U = A - \lambda_i E_n$.
Sei $q= min\{l \ | \ dim(Ker(U^l)) = dim(Ker(U^{l+1})\}$ und $s_j^{(i)} = dim(Ker(U^j))$ mit $j \leq q$.
Die Anzahl der Jordanbl\oee cke der Gr\oee sse $j$ zum Eigenwert $\lambda_i$ ist dann gegeben durch 
\\ \centerline{$s_j^{(i)} - s_{j-1}^{(i)} +s_{j+1}^{(i)}.$}
\\
\\
\\ \textit{Kommentar :} Die Anzahl der Jordanbl\oee cke zu $\lambda_i$ ist gegeben durch die geometrische Vielfachheit von $\lambda_i$.
\item Eine jordansche Normalform l\aee sst sich anhand dieser Parameter aufstellen.
\end{enumerate}
\end{remark}

\begin{remark}
\bt{Rechenverfahren zur Bestimmung der Basiswechselmatrix einer Jordan-Normalform} von $A \in M(n \times n; K)$
\begin{enumerate}
\item Bestimme die Eigenwerte $\lambda_1, ... \lambda_k$ der Matrix $A$. Das folgende Verfahren muss f\uee r jeden Eigenwert durchgef\uee hrt werdemn
\item Sei $U = A - \lambda_i E_n$.

Bestimme $Ker(U), ..., Ker(U^q)$ wobei $q= min\{l \ | \ Ker(U^l) = Ker(U^{l+1})\}$. 
\item Bestimme die Gr\oee sse der Jordanbl\oee cke des Eigenwerts $\lambda_i$.
\item F\uee r einen Jordanblock der Gr\oee sse $j$ bestimme einen Vektor $v_j^{(i)} \in Ker(U^q) / Ker(U^{q-1})$ zum Eigenwert $\lambda_i$. Dann ist $v_{k-1}{(i)} = (A - \lambda_i E_n)(v_k) $ f\uee r all $k = 1, ..., j-1.$
\item $\mathcal{B} = (v_1^{(1)}, ..., v_j^{(1)}, ..., v_1^{(k)}, ..., v_j^{(k)})$ ist die gesuchte Basis.

\end{enumerate}
\end{remark}

\section{Euklidische und unit\aee re Vektorr\aee ume}

\subsection{Das kanonische Skalarprodukt im $\mathbb{R}^n$}
\begin{definition}
Das \bt{kanonische Skalarprodukt} ist die Abbildung $\langle\quad, \quad\rangle: \mathbb{R}^{n} \times \mathbb{R}^{n} \rightarrow \mathbb{R}, \quad(x, y) \mapsto\langle x, y\rangle$ mit $\langle x, y\rangle = x^T \cdot y$.
\\Es hat die folgenden Eigenschaften:
\begin{enumerate}
%Bilinearitaet noch nicht komplett abgeschrieben
\item \bt{Bilinearit\aee t} $\left\langle \alpha x+ \alpha^{\prime} x^{\prime}, y\right\rangle=\alpha \langle x, y\rangle+ \alpha^{\prime}\left\langle x^{\prime}, y\right\rangle $
\\$\left\langle x, \beta y+ \beta^{\prime} y^{\prime}\right\rangle=\beta \langle x, y\rangle+ \beta^{\prime}\left\langle x, y^{\prime}\right\rangle $
\item \bt{Symmetrie} $\left\langle x, y \right\rangle = \left\langle y, x \right\rangle$
\item \bt{Positive Definitheit} $\left\langle x, x \right\rangle \geq 0$ und $\left\langle x, x \right\rangle = 0 \Leftrightarrow x = 0$.
\end{enumerate}
\end{definition}

\begin{definition}
Die \bt{euklidische Norm} ist die Abbildung $\|\|: \mathbb{R}^{n} \rightarrow \mathbb{R}_{+}, \quad x \mapsto\|x\|:=\sqrt{\langle x, x\rangle}$, mit dem kanonischen Skalarprodukt.
\\Dieses erf\uee llt folgende Eigenschaften
\begin{enumerate}
\item $\|x\| \geq 0$ und $\|x\| = 0 \Leftrightarrow x= 0$
\item $ \| \alpha \cdot x \| = \mid\alpha \mid \cdot \|x \| $ 
\item $\|x+y\| \leq\|x\|+\|y\|$
\end{enumerate}
\end{definition}

\begin{definition}
Mit Hilfe von $\langle\quad, \quad\rangle$ definiere \bt{Winkel} wie folgt:
\\Seien $x, y \in \mathbb{R}^n\backslash \{0\} $. Mit der Cauchy-Schwary-Ungleichung folgt: 
\begin{equation*}
-1 \leq \frac{\langle x, y\rangle}{\|x\| \cdot\|y\|} \leq+1.
\end{equation*}
Der Winkel zwischen $x$ und $y$ ist 
\begin{equation*}
\angle(x, y) = \arccos\left(\frac{\langle x, y\rangle}{\|x\| \cdot\|y\|}\right).
\end{equation*}
\end{definition}

\subsection{Das kanonische Skalarprodukt im $\mathbb{C}^n$}
\begin{definition}
Das \bt{kanonische Skalarprodukt im $\mathbb{C^n}$} analog zum kanonischen Skalarprodukt im $\mathbb{R^n}$ definiert. 
\\Es hat folgende Eigenschaften:
\begin{enumerate}
\item \bt{Sesquilinearit\aee t} $\left\langle \alpha x+ \alpha^{\prime} x^{\prime}, y\right\rangle_c=\alpha \langle x, y\rangle+ \alpha^{\prime}\left\langle x^{\prime}, y\right\rangle_c $
\\$\left\langle x, \beta y+ \beta^{\prime} y^{\prime}\right\rangle_c=\overline{\beta} \langle x, y\rangle+ \overline{\beta}^{\prime}\left\langle x, y^{\prime}\right\rangle_c $
\item \bt{Symmetrie} $\left\langle w, z \right\rangle_c = \overline{\left\langle z, w \right\rangle}_c$
\item \bt{Positive Definitheit} $\left\langle z, z \right\rangle \in \mathbb{R}_+$ und $\left\langle z, z\right\rangle_c = 0 \Leftrightarrow z = 0$
\end{enumerate}
\end{definition}


\subsection{Bilinearformen und Sesquilinearformen}
%Multilinearform und Bilinearform
\begin{definition}
Seien $V_1, V_2$ Vektor\aee ume und $K$ ein K\oee rper. Eine Abbildung
\begin{equation*}
s : V_1 \times V_2 \rightarrow K
\end{equation*}
heisst \bt{bilinear} falls
\begin{enumerate}
\item[\bt{B1}] $s( \alpha v_1 + \alpha^\prime v_1^\prime, v_2) = \alpha s(v_1, v_2) + \alpha^\prime s(v_1^\prime, v_2)$ und
\item[\bt{B2}] $s(v_1, \beta v_2 + \beta^\prime v_2^\prime) = \beta s(v_1, v_2) + \beta^\prime s(v_1, v_2^\prime)$.
\end{enumerate}
Man nennt einen bilineare Abbildung zus\aee tzlich \bt{symmetrisch}, wenn
\begin{enumerate}
\item[\bt{S}] $s(v, w) = s (w, v)$
\end{enumerate}
und \bt{alternierend}, falls
\begin{enumerate}
\item[\bt{A}] $s(w, v) = - s(v, w)$.
\end{enumerate}
\end{definition}

\begin{definition}
Eine \bt{Multilinearform} 
\begin{equation*}
s : V_1 \times V_2 \times ... \times V_n \rightarrow K
\end{equation*}
ist analog zur Bilinearform definiert, die Bilinearform also ein Spezialfall der Multilinearform.
\\Es gilt also f\uee r alle Koordinaten
\begin{enumerate}
\item[\bt{B1}] $s(v_1, ...,  \alpha v_k + \alpha^\prime v_k^\prime, ...,  v_n) = \alpha s(v_1, ..., v_k, ..., v_n) + \alpha^\prime s(v_1, ..., v_k^\prime, ..., v_n)$.
\end{enumerate}
\end{definition}



\begin{remark} Sei $V$ ein endlich dimensionaler Vektorraum, $s : V \times V \rightarrow K$ eine Bilinearform auf $V$ und $\mathcal{B} = (v_1, ..., v_n)$ eine geordnete Basis von $V$. Seien $v, w \in V$ und $x = \phi^{-1}_\mathcal{B}(v)$, $y = \phi^{-1}_\mathcal{B}(y)$, wobei $\phi^{-1}_B : \mathbb{K}^n \rightarrow V$ das Koordinatensystem von $\mathbb{B}$ ist. Dann ist
\begin{equation*}
s(v, w)= x^T A y=\left(x_{1}, \ldots, x_{n}\right)\left(\begin{array}{ccc}a_{11} & \cdots & a_{1 n} \\ \vdots & & \vdots \\ a_{n 1} & \cdots & a_{n n}\end{array}\right)\left(\begin{array}{c}y_{1} \\ \vdots \\ y_{n}\end{array}\right).
\end{equation*}
Es ist also $(a_{ij}) = s(v_i, v_j)$, wobei $i, j \in \{1, ..., n\}$.
\end{remark}

\begin{theorem}
Sei $V$ ein $ndimensionaler$ endlicher Vektorraum und $\mathcal{B}$ eine Basis von $V$. Dann ist  die Abbildung
\begin{equation*}
s \mapsto M_{\mathcal{B}}(s)
\end{equation*}
von den Bilinearformen auf $M(n \times n; K)$ bijektiv und $s$ genau dann symmetrisch, wenn $M_{\mathcal{B}}(s)$ symmetrisch. 
\end{theorem}

%eigentlich nicht wichtig da klar
\begin{lemma}
Seien $A, B \in M(n \ times n; K)$ mit 
\begin{equation*}
x_TAy = x_TBy = 
\end{equation*}
f\uee r alle $x, y \in K^n$. Dann ist $A = B$. 
\end{lemma}

%11032020
\begin{theorem}\bt{(Transformationformel)}
Sei $V$ eine endlichdimensionaler Vektorraum mit Basen $\mathcal{A}$ und $ \mathcal{B}$. F"ur jede Bilinearform $s$ auf $V$ gilt dann
\begin{equation*}
M_\mathcal{B}(s) =( T^\mathcal{B}_\mathcal{A})^T \cdot M_\mathcal{A}(s) \cdot T^\mathcal{B}_\mathcal{A}.
\end{equation*}
Der entscheidende Unterschied ist also $( T^\mathcal{B}_\mathcal{A})^T$ statt $( T^\mathcal{B}_\mathcal{A})^{-1}$.
\end{theorem}

\begin{definition}
Sei $s: V \times V \rightarrow K$ eine symmetrische Bilinearform. Die zu $s$ geh"orige quadratische Form ist die Abbildung
\begin{equation*}
q : V \rightarrow K, v \mapsto s(v, v).
\end{equation*}
Es gilt dann $g(\lambda v) = \lambda^2 q(v)$ f"ur  $\forall v \in V, \lambda \in K$.
Ist $V = K^n$ und $s$ durch die symmetrische Matrix $A = (a_ij)$ gegebene, gilt
\begin{equation*}
q(x_1, ..., x_n) = \sum\limits_{i, j = 1}^n a_{ij} x_i x_j = \sum\limits_{j = 1}^n a_ii x_i^2 + 2 \cdot \sum\limits_{1 \leq i < j \leq n} a_{ij}x_ix_j,
\end{equation*}
was eine homogenes quadratisches Polynom in den Unbestimmten $x_1, ..., x_n$ ist.
\end{definition}

\begin{theorem} \bt{(Polarisierung)}
Ist $\textrm{char}(K) \neq 2$, so gitl f"ur jede symmetrische Bilinearform $s$ und die zugeh"orige quadratische Form $q$:
\begin{equation*}
s(v, w) = \frac{1}{2}(q(v+w)-q(v)-q(w)) \text{ f"ur } \forall v, w \in V.
\end{equation*}
$s$ ist also aus $q$ konstruierbar.
\end{theorem}

%In Bermerkung und Definition aufteilen?
\begin{definition}
Eine Abbildung $F \in End(V)$ heisst \bt{semilinear}, falls 
\begin{equation*}
F(v + \lambda w) = F(v ) + \overline{\lambda} F(w) \text{ f"ur } \forall v, w \in V, \lambda \in \mathbb{C}.
\end{equation*}
Man nennt eine Abbildung $s: V \times V \rightarrow \mathcal{C}$ \bt{sesquilinear}, wenn $s$ im ersten Argument linear und im zweiten Element semilinear ist.

Ist $\mathcal{A}$ eine Basis von $V$, so ist die Matrix der Sesquilinearform $s : V \times V \rightarrow \mathbb{C}$
\begin{equation*}
A = M_\mathcal{A}(s) = (s(v_i, v_j)).
\end{equation*}
Ist $v = \phi_\mathcal{A}(x), w = \phi_\mathcal{A} (y)$ so ist 
\begin{equation*}
s(v, w) = x^T A \overline{y},
\end{equation*}
wobei $\overline{y}$ die komponentenweise Konjugation ist.
\\
\\


Ist \(\mathcal{B}\) eine weitere Basis, \(B=M_{\mathcal{B}}(s)\) und \(T=T_{\mathcal{A}}^{\mathcal{B}},\) so hat man die \bt{Transformationsformel}
\begin{equation*}
B= T^T A \overline{T}.
\end{equation*}

Man hat im Komplexen für Sesquilinearformen eine \bt{Polarisierung}
\begin{equation*}
s(v, w)=\frac{1}{4}(q(v+w)-q(v-w)+\mathrm{i} q(v+\mathrm{i} w)-\mathrm{i} q(v-\mathrm{i} w)).
\end{equation*}
\end{definition}

\begin{definition}
$s : V \times V \rightarrow K$ heisst ausserdem \bt{hermitesch}, wenn gilt
\begin{equation*}
s(v, w) = \overline{s(w, v)}.
\end{equation*}
$s$ ist hermetisch also genau dann, wenn f"ur eine Basis $\mathcal{A}$ gilt
\begin{equation*}
M_\mathcal{A}(s) = \overline{M_\mathcal{A}(s)}.
\end{equation*}
\end{definition}

%richtiges Zeichen ? zwischen UVR und Komplement unterscheiden.
\begin{definition}
Sei $K = \mathbb{R}, \mathbb{C}$ und $V$ eine K-Vektorraum mit einer symmetrischen Bilinearform / hermiteschen Sesquilinearform $s : V \times V \rightarrow K$. 
\\ Dann heisst $s$ \bt{positiv definit}, falls $s(v, v) > 0 \ \forall v \in V/\{0\}$.
Eine solche Bilinearform / Sesquilinearform nennt man auch \bt{Skalarprodukt auf $V$}.
\\$V$ zusammenn mit einem Skalarprodukt nennt man einen \bt{euklidischen Vektorraum} ($K = \mathbb{R})$) bzw. \bt{unit"aren Vektorraum} ($K = \mathbb{C}$).
\\Die \bt{Norm} auf einem euklidischen / unit"aren Vektorraum ist gegeben durch
\begin{equation*}
\\
\| v\| = \sqrt{\left\langle v , v \right\rangle}.
\end{equation*}
\end{definition}

\begin{remark}
Eines symmetrische (bzw. hermitesche) \bt{Matrix} $A \in M( n \times n; K)$ heisst \bt{positiv definit}, falls die zugeh"orige Bilinearform / Sesquilinearform positiv definit ist. 
\end{remark}

\begin{theorem} \bt{(Ungleichung von Cauchy-Schwarz)} Ist $V$ ein euklidischer Vektorraum, gilt f"ur alle $v, w \in V$
\begin{equation*}
|\langle v, w\rangle| \leq\|v\| \cdot\|w\|
\end{equation*}
und Gleichheit genau dann, wenn $v$ und $w$ linear abh"angig sind.
\end{theorem}

\begin{definition}
Definition. Sei \(V\) ein euklidischer bzw. unitärer Vektorraum.
\begin{enumerate}
\item Zwei Vektoren \(v, w \in V\) heißen orthogonal ($v \perp w$), wenn
\begin{equation*}
\langle v, w\rangle= 0.
\end{equation*}
\item Zwei Untervektorräume \(U, W \subset V\) heißen orthogonal ($U \perp W$), wenn
\begin{equation*}
u \perp w \text { für alle } u \in U \text { und alle } w \in W \text {. }
\end{equation*}
\item Ist \(U \subset V\) ein Untervektorraum, so definiert man sein orthogonales Komplement
\begin{equation*}
U^{\perp}:=\{v \in V: v \perp u \text { für alle } u \in U\}
\end{equation*}
Das ist wieder ein Untervektorraum.
\item Eine Familie \(\left(v_{1}, \ldots, v_{n}\right)\) in \(V\) heißt orthogonal, wenn
\begin{equation*}
v_{i} \perp v_{j} \quad \text { für alle } i \neq j.
\end{equation*}
Sie heißt orthonormal, falls zusätzlich
\begin{equation*}
\|v_i\| = 1 \text{ f"ur alle } i.
\end{equation*}
Sie heisst Orthonormalbasis, falls sie auch eine Basis ist, d.h. eine Basis mit
\begin{equation*}
\left\langle v_{i}, v_{j}\right\rangle=\delta_{i j} \ \forall i, j.
\end{equation*}
%orthogonal zeichen statt oplus
\item Ist \(V=V_{1} \oplus \ldots \oplus V_{k},\) so heißt die direkte Summe orthogonal ($V=V_{1} \obot \ldots \obot V_{k}$), falls
\begin{equation*}
\text { falls } V_{i} \perp V_{j} \quad \text { für alle } i \neq j.
\end{equation*}
\end{enumerate}
\end{definition}

\begin{remark}
Sei $(v_1, ..., v_n)$ eine orthogonale Familie in $V$ mit $v_i \neq 0$ $\forall i$. Dann gilt:
\begin{enumerate}
\item Dann ist 
$$
\left(\frac{1}{\|v_1\|} v_1, \frac{1}{\|v_2\|} v_2, ..., \frac{1}{\|v_n\|} v_n \right)
$$
orthonormal.
\item $(v_1, ..., v_n)$ ist linear unabh"angig.
\end{enumerate}
\end{remark}

\begin{remark}
Sei $(v_1, ..., v_n)$ eine Orthonormalbasis von $V$ und $v \in V$ beliebig. Dann gilt:
$$
v = \left\langle v, v_1 \right\rangle v_1 + ... + \left\langle v, v_n \right\rangle v_n.
$$
\end{remark}

\begin{remark}
Sei $(v_1, ..., v_n)$ eine Orthonormalbasis von $V$ und $v = \lambda_1 v_1 + ... + \lambda_n v_n, \ w =  \mu v_1 + ... + \mu v_n \  \in V$. Dann gilt:
$$
\left\langle v, w \right\rangle = \left\langle \lambda_1 v_1 + ... + \lambda_n v_n, \mu_1 v_1 + ... + \mu_n v_n \right\rangle = \sum\limits_{i = 1}^n \lambda_i \overline{\mu_j} = \left(\begin{array}{ccc}\lambda_1 & \hdots & \lambda_n \end{array} \right) \overline{\left(\begin{array}{c}\mu_1  \\ \vdots \\ \mu_n \end{array} \right)}.
$$
\end{remark}

\subsection{Gram-Schmidtsches Orthogonalisierungsverfahren}
\begin{theorem} \bt{(Orthonormalisierungssatz)}
Sei $V$ ein endlichdimensionaler, euklidischer bzw. unit"arer Vektorraum und $W \subset V$ ein Untervektorraum mit der Orthonormalbasis $(w_1, ..., w_m)$. Dann gibt es eine Erg"anzung zu einer Orthonormalbasis von $V$ $(w_1, ..., w_n, w_{m+1}, ..., w_n)$.
\end{theorem}

\begin{corollary}
Jeder endlichdimensionale Vektorraum hat eine Orthonormalbasis.
\end{corollary}

\begin{corollary}
Sei $W \subset V$ ein Untervektorraum des endlich dimensionalen euklidischen bzw. unit"aren Vektorraumes $V$. Dann gilt
$$
V = W \obot W^\perp \text{ und } dim V = dim W + dim W^\perp.
$$
\end{corollary}

\begin{remark}\bt{(Gram-Schmidtsches Orthogonalisierungsverfahren)}
\begin{enumerate}
\item Erg"anze $(w_1, ..., w_n)$ zu einer Basis $(w_1, ..., w_m, v_{m+1}, ..., v_n)$ von $V$
\item Orthogonalisiere die Basis durch $\tilde{v_i} := v_i - \left( \scp{v_i}{w_1}  w_1 +  ...  + \scp{v_i}{w_m} w_m  \right)$ 
\item Normiere die erhaltenen Vektoren  durch $v_i' = \frac{\tilde{v_i}}{\|\tilde{v_i}\|}.$
\end{enumerate}
\end{remark}

\begin{remark} \bt{(QR-Zerlegung)}
Sei $V = K^n$ und $A \in M(n \times n, K)$. Dann existiert eine Orthogonalmatrix  $Q$ und eine obere Dreiecksmatrix $R$, so dass $$
A = Q \cdot R.$$
\end{remark}

\subsection{Gramsche Determinante und Volumenmessung}

\begin{definition}
Seien $V$ ein euklidischer Vektorraum und $v_1, ..., v_m \in V$, wobei $m \leq n = dim V$.
Dann nennt man 
$$
G\left(v_{1}, \ldots, v_{m}\right):=\operatorname{det}\left(\begin{array}{ccc}\left\langle v_{1}, v_{1}\right\rangle & \cdots & \left\langle v_{1}, v_{m}\right\rangle \\ \vdots & & \vdots \\ \left\langle v_{m}, v_{1}\right\rangle & \cdots & \left\langle v_{m}, v_{m}\right\rangle\end{array}\right) \in \mathbb{R}
$$
die Gramsche Determinante.
\end{definition}

\begin{remark}
Es ist $G\left(v_{1}, \ldots, v_{m}\right) \geq 0$ und es gilt
$$
G\left(v_{1}, \ldots, v_{m}\right) > 0 \Leftrightarrow v_1, ..., v_m \text{ sind linear unabh"angig.}
$$
\end{remark}

\begin{definition}
Das Volumen des von  $v_1, ..., v_m$ aufgespannten Spates ist definiert als 
$$
Vol(v_1, ..., v_m) := \sqrt{G\left(v_{1}, \ldots, v_{m}\right)}.
$$
\end{definition}

\begin{theorem}\bt{Ungleichung von Hadamard}
Seien $v_1, ..., v_m \in V$ und $V$ ein euklidischer Vektorraum, wobei $m \leq dim V$. Dann gilt
$$
Vol(v_1, ..., v_m) \leq \|v_1\| \cdot ... \cdot \|v_m\|.
$$
Es besteht Gleichheit genau dann, wenn $v_i \perp v_j$ f"ur $\forall i \neq j$.


\end{theorem}


\section{N\uee tzliche Formeln und Hinweise}

\end{document}
