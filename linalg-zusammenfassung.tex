%%% DOCUMENT TYPE %%%%%%%%%%%%%%%%%%%%%%%%%%%%%%%%%%%%%%%%%%%%%%%%%%%%%%%%%%%%%%

\documentclass[11pt, a4paper]{article}

%%% SETUP %%%%%%%%%%%%%%%%%%%%%%%%%%%%%%%%%%%%%%%%%%%%%%%%%%%%%%%%%%%%%%%%%%%%%%

% Analysis


\usepackage{mathtools}
\usepackage{amsfonts} 
\usepackage{fancybox}
\usepackage{textcomp }
\usepackage{bbold}
\usepackage{amssymb}
\usepackage{titlesec}
\usepackage[english,ngerman]{babel}

\title{Lineare Algebra II}
\author{Frederic Jørgensen}
\date{Erstellt: 01. Februar 2020, Zuletzt ge\aee ndert: \today \\                           % deutsche Ausgabe
	 \selectlanguage{ngerman}}
\pagestyle{headings}
\setlength{\parindent}{0pt}


\newcommand{\sectionbreak}{\clearpage} %Seite bei jedem Abschnitt wechs^{•}eln


%%% PACKAGES %%%%%%%%%%%%%%%%%%%%%%%%%%%%%%%%%%%%%%%%%%%%%%%%%%%%%%%%%%%%%%%%%%%

% Encoding

\usepackage[utf8]{inputenc}
\usepackage[T1]{fontenc}

% Geometry

\usepackage{geometry} % edit margins of paper
\usepackage{setspace} % edit line spacing
\usepackage{fancyhdr} % header, footer
\usepackage{titlesec} % edit format of titles

% Visual

\usepackage[dvipsnames]{xcolor} % colors
\usepackage{tikz} % graphics
\usepackage[framemethod=tikz]{mdframed} % frames, better theorems

% Math

\usepackage{amsmath} % math tools
\usepackage{amssymb} % math symbols
\usepackage{amsthm} % thereoms
\usepackage{mathtools} % math tools

% Referencing

\usepackage{nameref}
\usepackage{hyperref}
\usepackage{cleveref}

% Useful

\usepackage[shortlabels]{enumitem} % enumerations

% Other

\usepackage{lastpage} % get number of last page

%%% MARGINS %%%%%%%%%%%%%%%%%%%%%%%%%%%%%%%%%%%%%%%%%%%%%%%%%%%%%%%%%%%%%%%%%%%%

\geometry{a4paper, left=20mm, right=20mm, top=20mm, bottom=20mm, includehead}

%%% COLORS %%%%%%%%%%%%%%%%%%%%%%%%%%%%%%%%%%%%%%%%%%%%%%%%%%%%%%%%%%%%%%%%%%%%%

%%% COLOR DEFINITIONS %%%%%%%%%%%%%%%%%%%%%%%%%%%%%%%%%%%%%%%%%%%%%%%%%%%%%%%%%%

\colorlet{color-definition}              {Blue!20}%{SpringGreen!20}
\colorlet{color-theorem}                {Brown!25}%{Apricot!13}
\colorlet{color-proposition}            {ProcessBlue!13}% {Apricot!13}
\colorlet{color-corollary}              {Salmon!12}%{Apricot!13}
\colorlet{color-lemma}                  {Brown!7}%{Apricot!13}
\colorlet{color-remark}                 {Gray!4}
\colorlet{color-example}                {Lavender!7}
% \colorlet{color-proof}                  {FILL COLOR HERE}


%%% CAPTIONS %%%%%%%%%%%%%%%%%%%%%%%%%%%%%%%%%%%%%%%%%%%%%%%%%%%%%%%%%%%%%%%%%%%

%%% CAPTION DEFINITION %%%%%%%%%%%%%%%%%%%%%%%%%%%%%%%%%%%%%%%%%%%%%%%%%%%%%%%%%

\newcommand*{\definitionname}{Definition}
\newcommand*{\theoremname}{Theorem}
\newcommand*{\propositionname}{Proposition}
\newcommand*{\corollaryname}{Corollary}
\newcommand*{\lemmaname}{Lemma}
\newcommand*{\remarkname}{Remark}
\newcommand*{\examplename}{Example}


%%% LANGUAGE %%%%%%%%%%%%%%%%%%%%%%%%%%%%%%%%%%%%%%%%%%%%%%%%%%%%%%%%%%%%%%%%%%%

% load language setup (may also include loading packages)

%%% SETUP %%%%%%%%%%%%%%%%%%%%%%%%%%%%%%%%%%%%%%%%%%%%%%%%%%%%%%%%%%%%%%%%%%%%%%

\usepackage[english]{babel}

%%% CAPTION REDEFINITION %%%%%%%%%%%%%%%%%%%%%%%%%%%%%%%%%%%%%%%%%%%%%%%%%%%%%%%



%%% HYPHENATION %%%%%%%%%%%%%%%%%%%%%%%%%%%%%%%%%%%%%%%%%%%%%%%%%%%%%%%%%%%%%%%%



%%% TITLES %%%%%%%%%%%%%%%%%%%%%%%%%%%%%%%%%%%%%%%%%%%%%%%%%%%%%%%%%%%%%%%%%%%%%

\setcounter{secnumdepth}{2}

\titleformat{\section}[block]
{\normalfont\Large\bfseries}{\thesection}{1em}{}
\titleformat{\subsection}[block]
{\normalfont\large\bfseries}{\thesubsection}{1em}{}
\titleformat{\subsubsection}[block]
{\normalfont\normalsize\bfseries}{\thesubsubsection}{1em}{}

\titlespacing*{\section}{0pt}{3.25ex plus 1ex minus .2ex}{2.0ex plus .2ex}
\titlespacing*{\subsection}{0pt}{2.75ex plus 1ex minus .2ex}{1.0ex plus .2ex}
\titlespacing*{\subsubsection}{0pt}{2.25ex plus 0.8ex minus 0.1ex}{0.0ex plus .2ex}

%%% SPACING, LENGTHS, INDENTATION %%%%%%%%%%%%%%%%%%%%%%%%%%%%%%%%%%%%%%%%%%%%%%

\setstretch{1.05} % scaling of space between lines
% \setlength{\topsep}{FILL LENGTH HERE}
% \setlength{\itemsep}{FILL LENGTH HERE}
\setlength{\parskip}{4.0pt plus 1.0pt minus 1.0pt} % space between paragraphs
\setlength{\parindent}{0pt} % indentation of paragraphs

%%% HEADER, FOOTER %%%%%%%%%%%%%%%%%%%%%%%%%%%%%%%%%%%%%%%%%%%%%%%%%%%%%%%%%%%%%

\pagestyle{fancy}
\fancyhf{} % clear everything
\lhead{}
\chead{\large \bfseries \LaTeX{} (Math) Template}
\rhead{Page \thepage /\pageref*{LastPage}}
\lfoot{}
\cfoot{}
\rfoot{}

%%% SHORTCUTS %%%%%%%%%%%%%%%%%%%%%%%%%%%%%%%%%%%%%%%%%%%%%%%%%%%%%%%%%%%%%%%%%%

%%% SINGLE SYMBOLS %%%%%%%%%%%%%%%%%%%%%%%%%%%%%%%%%%%%%%%%%%%%%%%%%%%%%%%%%%%%

% Logic

% \forall exists
% \exists exists
% \lnot exists
% \lor exists
% \land exists
\newcommand*{\limp}{\rightarrow}
\newcommand*{\limps}{\; \limp \;} % \limp with some space around
\newcommand*{\leqv}{\leftrightarrow}
\newcommand*{\leqvs}{\; \leqvs \;} % \leqv with some space around

% Meta Logic

% \implies exists
% \iff exists

% Colon Stuff

\newcommand*{\cl}{\colon}
\newcommand*{\cleq}{\coloneqq}
\newcommand*{\eqcl}{\eqqcolon}

% Sets

\newcommand*{\N}{\mathbb{N}} % natural numbers
\newcommand*{\Z}{\mathbb{Z}} % integers
\newcommand*{\Q}{\mathbb{Q}} % rational numbers
\newcommand*{\R}{\mathbb{R}} % real numbers
\newcommand*{\C}{\mathbb{C}} % complex numbers

%%% MATH OPERATORS %%%%%%%%%%%%%%%%%%%%%%%%%%%%%%%%%%%%%%%%%%%%%%%%%%%%%%%%%%%%%

% General

\DeclareMathOperator{\id}{id}
\DeclareMathOperator{\sgn}{sgn}

%%% TEMPLATES %%%%%%%%%%%%%%%%%%%%%%%%%%%%%%%%%%%%%%%%%%%%%%%%%%%%%%%%%%%%%%%%%%

% General

% write a set definition like: { #1 | #2 }
\newcommand*{\setdefinition}[2]{
  \{#1 \mid #2\}
}

% write a nice map definition
\newcommand*{\mapdefinition}[5]{
  \begin{align*}
    #1 \cl #2 &\to     #3 \\
           #4 &\mapsto #5
  \end{align*}
}


%%% FORMATTING %%%%%%%%%%%%%%%%%%%%%%%%%%%%%%%%%%%%%%%%%%%%%%%%%%%%%%%%%%%%%%%%%

%%% SYMBOLS USED BY NUMBERINGS, ENVIRONMENTS, ... %%%%%%%%%%%%%%%%%%%%%%%%%%%%%%

% \renewcommand*\qedsymbol{$\blacksquare$} % alternative QED symbol
\renewcommand{\thefootnote}{\arabic{footnote}} % normal footnotes on page
\renewcommand{\thempfootnote}{\fnsymbol{mpfootnote}} % footnotes on minipages, e.g. in mdframed environments

%%% MDFRAMED STYLES %%%%%%%%%%%%%%%%%%%%%%%%%%%%%%%%%%%%%%%%%%%%%%%%%%%%%%%%%%%%

% thick frame and bar for title

\mdfdefinestyle{style-box}{
  linewidth=1pt,
  linecolor=Gray!20,
%   roundcorner=3pt,
  innerleftmargin=0.4\baselineskip,
  innerrightmargin=0.2\baselineskip,
  innertopmargin=0.2\baselineskip,
  innerbottommargin=0.2\baselineskip,
  frametitlebackgroundcolor=Gray!20,
  frametitleaboveskip=0.2pt,
  frametitlebelowskip=0.2pt,
  theoremseparator=,
  theoremspace=\hfill,
  theoremtitlefont=\mdseries\scshape,
  nobreak=true
}

% highlighted background

\mdfdefinestyle{style-background}{
  hidealllines=true,
  backgroundcolor=Gray!5,
  innerleftmargin=0.4\baselineskip,
  innerrightmargin=0.2\baselineskip,
  innertopmargin=0.2\baselineskip,
  innerbottommargin=0.2\baselineskip,
}

% thin frame

\mdfdefinestyle{style-thinframe}{
  linewidth=0.4pt,
  linecolor=Gray!30,
  innerleftmargin=0.5\baselineskip,
  innerrightmargin=0.5\baselineskip,
  innertopmargin=0.4\baselineskip,
  innerbottommargin=0.4\baselineskip,
}

%%% ENVIRONMENTS %%%%%%%%%%%%%%%%%%%%%%%%%%%%%%%%%%%%%%%%%%%%%%%%%%%%%%%%%%%%%%%

% Definition

\mdtheorem[
  style=style-box,
  linecolor=color-definition,
  frametitlebackgroundcolor=color-definition
]{definition}{\definitionname}[section]

% Theorem

\mdtheorem[
  style=style-box,
  linecolor=color-theorem,
  frametitlebackgroundcolor=color-theorem,
  font=\itshape
]{theorem}{\theoremname}[section]

% Proposition

\mdtheorem[
  style=style-box,
  linecolor=color-proposition,
  frametitlebackgroundcolor=color-proposition,
  font=\itshape
]{proposition}[theorem]{\propositionname}

% Corollary

\mdtheorem[
  style=style-box,
  linecolor=color-corollary,
  frametitlebackgroundcolor=color-corollary,
  font=\itshape
]{corollary}[theorem]{\corollaryname}

% Lemma

\mdtheorem[
  style=style-box,
  linecolor=color-lemma,
  frametitlebackgroundcolor=color-lemma,
  font=\itshape
]{lemma}[theorem]{\lemmaname}

\theoremstyle{remark}

% Remark

\newtheorem*{remark}{\remarkname}
\surroundwithmdframed[
  style=style-background,
  backgroundcolor=color-remark
]{remark}

% Enumeration Remark

\newenvironment{enumremark}{
  \begin{remark}
    \begin{enumerate}[(a)]
      \item[]
}{
    \end{enumerate}
  \end{remark}
}

% Example

\newtheorem*{example}{\examplename}
\surroundwithmdframed[
  style=style-background,
  backgroundcolor=color-example
]{example}

% Proof

\surroundwithmdframed[
  style=style-thinframe
]{proof}

%%% TEXT FORMATTING %%%%%%%%%%%%%%%%%%%%%%%%%%%%%%%%%%%%%%%%%%%%%%%%%%%%%%%%%%%%

% definitions

\newcommand*{\df}[1]{\colorbox{color-definition}{\emph{#1}}}



%%% DOCUMENT %%%%%%%%%%%%%%%%%%%%%%%%%%%%%%%%%%%%%%%%%%%%%%%%%%%%%%%%%%%%%%%%%%%

%%Todo:
% Definition Integritaetsring


\begin{document}
\maketitle
\tableofcontents
\newpage
\begin{abstract}
	%enterlater
	\centerline{\bt{Stand zum ersten Semester}, Template von Johann Birnick}
\end{abstract}
\section{Eigenwerte}
\subsection{Trigonalisierung}
\begin{definition}
Sei $F : V\ \rightarrow V$ ein Endomorphismus und $W \subset V$ ein Untervektorraum. $W$ heisst $F$-invariant, wenn $F(W) \subset W$.
\end{definition}
\begin{remark}
Ist $W \subset V$ ein $F$-invarianter Unterraum, so ist $P_{F|W}$ ein Teiler von $P_F$.
\end{remark}

\begin{definition}
Eine \bt{Fahne} $(V_r)$ in einem $n$-dimensionalem $V$ ist eine Kette 
\\ \centerline{ $\{0\} = V_0 \subset V_1 \subset ... \subset V_n = V$}
von Untervektorr\aee umen mit $dim V_r = r$. Ist $F \in End(V)$, so heisst die Fahne $F$-invariant, wenn 
\\ \centerline{$F(V_r) \subset V_r$ f\uee r alle $r \in \{0, ... n\}$.}
\end{definition}

\begin{remark}
F\uee r $F \in End(V)$ sind folgende Bedingungen \aee quivalent:
\begin{enumerate}
\item[(i)] Es gibt eine $F$-invariante Fahne in $V$.
\item[(ii)] Es gibt eine Basis $\mathcal{B}$ von V, so dass $M_\mathcal{B}  (F)$ eine obere Dreiecksmatrix ist.
\end{enumerate}
\end{remark}


\begin{theorem} \textbf{(Trigonalisierungssatz)} F\uee r einen Endomorphismus $F$ eines $n$-dimensionalen $K$-Vektorraumes sind folgende Bedingungen \aee quivalent:
\begin{enumerate}
\item[(i)] $F$ ist trigonalisierbar.
\item[(ii)] Das charakteristische Polynom $P_F$ zerf\aee llt in Linearfaktoren.
\end{enumerate}
\end{theorem}

\begin{corollary} Jeder Endomorphismus eines endlich-dimensionalen komplexen Vektorraumes ist trigonalisierbar.
\end{corollary}

\begin{remark} \bt{Rechenverfahren zur Trigonalisierung eines Endomorphismus} von $A \in M(n \times n; K)$
\begin{enumerate}
\item Pr\uee fe, dass $P_A$ in Linearfaktoren zerf\aee llt, aber A nicht diagonalisierbar ist. 
\\
Bestimme einen Eigenvektor $v_1 \in Eig(A, \lambda_1 )$ und erg\aee nze diesen zu einer Basis $\mathcal{B} := (v_1, e_i, ..., e_j)$ des $K^n$. 
\\Betrachte $S^{-1}_1 := T^{\mathcal{B}_1}_{\mathcal{K}}$ und berechne $A_2 = S_1 \cdot A \cdot S_1^{-1} = $
$\left(\begin{array}{ccccc}{
\lambda_{1}} & {*} & {\cdots} & {\cdots} & {*} 
\\ {0} & {\lambda_{2}} & {*} & {\cdots} & {*} 
\\ {\vdots} & {0} & {} & {} 
\\ {\vdots} & {\vdots} & {} & {A_{2}^{\prime}} 
\\ {0} & {0} & {} & {} & {}\end{array}\right)$
\item Bestimme einen Eigenvektor $v_2^{\prime} \in Eig(A_{2}^{\prime}, \lambda_2) $ f\uee r einen Eigenwert $\lambda_2$ von $A_{2}^{\prime}$, erg\aee nze $v_1, v_2$ zu einer Basis $\mathcal{B}_2$ des $K^n$, wobei $v_2$ wie  $v_2^{\prime}$ ist, aber zus\aee tzlich eine 0 als ersten Eintrag hat. Betrachte $S^{-1}_2 := T^{\mathcal{B}_2}_{\mathcal{K}}$ und berechne $A_3 = S_2 \cdot A \cdot S_2^{-1} = $
$\left(\begin{array}{ccccc}{
\lambda_{1}} & {*} & {\cdots} & {\cdots} & {*} 
\\ {0} & {\lambda_{2}} & {*} & {\cdots} & {*} 
\\ {\vdots} & {0} & {} & {} 
\\ {\vdots} & {\vdots} & {} & {A_{3}^{\prime}} 
\\ {0} & {0} & {} & {} & {}\end{array}\right)$

\item Das macht so oft, bis $ D := A_n =S_{n-1} \cdot A \cdot S_{n-1}^{-1}$ eine obere Dreiecksmatrix ist.
\end{enumerate}
\end{remark}
\subsection{Potenzen eines Endomorphismus}
%Anderes aus Linalg I ergaenzen?
\begin{definition}
$F \in End(V)$ heisst \bt{nilpotent}, wenn $F^k = 0$ f\uee r ein $k \geq 1$.
\end{definition}
\begin{theorem}
Ist $F \in End(V)$ und $n = dim(V)$. Dann sind folgende Aussagen \aee quivalent:
\begin{itemize}
\item[(i)] $F$ ist nilpotent.
\item[(ii)] $F^d = 0$ f\uee r ein d mit $1 \leq d \leq n$.
\item[(iii)] $P_F = \pm t^n$.
\item[(iv)] Es gibt eine Basis $\mathcal{B}$ von $V$, so dass 
\\ \centerline{
$M_{\mathcal{B}}(F)=\left(\begin{array}{ccc}
{0} & {} & {*} \\ {} & {\ddots} & {} 
\\ {0} & {} & {0}\end{array}\right)$.
}
\end{itemize}
\end{theorem}

\subsection{Die Jordansche Normalform}
\begin{definition}
Sei $F \in End(V)$, so dass $P_F$ in Linearfaktoren zerf\aee llt. Wenn $dim(Eig(F; \lambda_i)) < \mu (P_F; \lambda_i) =: r_i$, kann man die Dimension des Eigenraums vergr\oee ssern: 
\\ \centerline{$Eig(F; \lambda_i) \subset Ker(F - \lambda_i id_v)^{r_i} = Hau(F; \lambda_i)$. }
$Hau(F; \lambda_i)$ nennt man den \bt{Hauptraum} von $F$ zum Eigenwert $\lambda_i$.
\end{definition}
\begin{theorem} \bt{(Satz \uee ber die Hauptraumzerlegung)}
Sei $F \in End_K(V)$, so dass $P_F$ in Linearfaktoren zerf\aee llt. Es sei $V_i := Hau(F, \lambda_i)$ f\uee r alle paarweise verschiedenen Eigenwerte $\lambda_1, ... , \lambda_n \in K$ von $F$. Sei $k := Rg(F)$. Dann gilt:
\begin{enumerate}
\item $F(V_i) \subset V_i$ und $dim(V_i) = \mu (P_F, \lambda_i)$ f\uee r $i = 1, ..., k$. 
\item $V =\bigoplus\limits_{i \in I} V_i$ mit $I=1, ..., k$.
\item $F$ hat eine Zerlegung $F = F_D + F_N$ mit
\begin{enumerate}
\item $F_D$ ist diagonalisierbar
\item$ F_N$ ist nilpotent
\item $F_N$ und $F_D$ kommutieren
\item $F_N$ und $F_D$ lassen sich als Polynome von $F$ schreiben und kommutieren insbesondere mit $F$
\item Wenn man (a), (b) und (c) verlangt, ist diese Zerlegung eindeutig
\end{enumerate}
\end{enumerate}
\end{theorem}

\begin{corollary}
Sei $A \in M(n \times n; K)$, so dass $P_A$ in Linearfaktoren zerf\aee llt. Dann $\exists S \in GL(n)$: 
$$
S A S^{-1}=\left(\begin{array}{cccc}{} & {} & {} & {} 
\\ {\lambda_{1} E_{r_{1}}+N_{1}} & {} & {} & {0} 
\\ {} & {} & {\ddots} & {} 
\\ {0} & {} & {} & {\lambda_{1} E_{r_{1}}+N_{1}} 
\end{array}\right)
$$
wobei f\uee r $i = 1, ... k$
\\ \centerline{$
\lambda_{1} E_{r_{1}}+N_{1} = 
\left(\begin{array}{ccc}
{\lambda_{i}} & {} & {*} 
\\ {} & {\ddots} & {} 
\\ {0} & {} & {\lambda_{i}}\end{array}\right) \in M(r_i \times r_i; K)$.}
mit nilpotenten $N_i$. Insbesondere ist $\tilde{A} = D + N$, wobei $D$ eine Diagonalmatrix und $N$ nilpotent ist und $N$ und $D$ kommutieren.
%%%
\end{corollary}

\begin{lemma}
\bt{(Lemma von Fitting)} Zu einem $G \in End_K(V) $ betrachte
\\ \centerline{$d := min \{ l \in \N : Ker (G^l) =Ker(G^{l + 1}) \}$ und $r := \mu (P_G ; 0)$.}
\\wobei $G^0 := id_V$. Dann gilt:
\begin{enumerate}
\item $d =min \{l : Im(G^l) = Im(G^{l + 1}) \}.$
\item $Ker (G^{d + i}) = Ker (G), \ Im (G^{d + i}) = Im (G^d)\ \forall i \in \N$.
\item Die R\aee ume $U := Ker (G^d)$ und $W := Im (G^d)$ sind $G$-invariant.
\item $(G|U)^d = 0$ und $G|W: W \rightarrow W$ ist ein Isomorphismus.
\item F\uee r das Minimalpolynom von $G|U$ gilt $M_{G|U} = t^d$.
\item $V = U \oplus W$, $dim(U) = r \geq d$, $dim(W) = n - r$.
\\ Insbesondere gibt es eine Basis $\mathcal{B}$ von $V$, so dass
\\ \centerline{$M_\mathcal{B}(G) =
\left(\begin{array}{ll}{N} & {0} 
\\ {0} & {C}\end{array}\right) 
$
mit  $N^d = 0$ und $C \in GL (n-r; K)$.}
\end{enumerate}
\end{lemma}
%to be continued
%Satz seite 256 nochmal aufschreiben

\section{N\uee tzliche Formeln und Hinweise}
\end{document}
